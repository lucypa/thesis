\chapter{Implementation}\label{ch:implementation}

Prior to this thesis, the seL4 Device Driver Framework already supported a minimal, single client
networking system. This included an Ethernet driver, layer-one Rx multiplexer (limited to a single 
client), copy component, a Tx multiplexer (also limited to a single client and containing no policy) and 
a simple echo server client application. To properly support multiple client applications, we
extend the Rx multiplexer to multiplex to multiple different client applications, and implement 3
new transmit multiplexers. We also implement a simple timer driver to interface with our 
transmit multiplexer when required, and finally, a new ARP component to handle broadcast traffic. 
We configure all queues between these components to further implement different
policies in \autoref{ch:evaluation}.

\section{Transmit Multiplexer Policies}

We implement 3 different transmit multiplexers, each implementing a different policy. 
We present our pseudo code designs for how each multiplexer process transmit packets from
the client. All multiplexers return free buffers from the driver to their respective 
clients in the same FIFO order and hence this function has been omitted for brevity. 
Notably, the implementations are very simple, aiding any debugging effort required, but also
easing future formal verification endeavours.

\subsection{Round-robin Policy}
We implement a round-robin policy by processing a single client's transmit request at a time.
This is done in a loop to ensure we process any batched requests as outlined in \autoref{l:roundrobin}.

\begin{figure} [H]
    \begin{minted}[fontsize=\small]{c}
void process_transmit_ready() {
  enqueued = 0;
  old_enqueued = 0;

  while(!ring_full(driver->used_ring)) {
    old_enqueued = enqueued;
    for each client {
      /* Process a single used buffer at a time */
      if (!ring_empty(client->used_ring) &&
          !ring_full(driver->used_ring)) {
        
        buf = dequeue(client->used_ring);
        phys = get_phys_addr(buf);
        enqueue(driver->used_ring, phys);
        enqueued += 1;
      }
    }
      /* we haven't processed any packets since 
        last loop, exit */
    if (old_enqueued == enqueued) break;
  }
}
\end{minted}
\caption{Transmit round-robin Policy}
\label{l:roundrobin}
\end{figure}

\subsection{Priority-based Policy}

We implement a priority-based policy in \autoref{l:prioritybased}. We assume the list of
clients is ordered from highest priority to lowest. Thus it is possible to change a clients
priority at run time by manipulating this list.\\

\begin{figure} [H]
    \begin{minted}[fontsize=\small]{c}
void process_transmit_ready() {
  enqueued = 0;
  old_enqueued = 0;
  while(!ring_full(driver->used_ring)) {
    old_enqueued = enqueued;
    for each client {
      while(!ring_empty(client->used_ring) &&
            !ring_full(driver->used_ring)) {

        buf = dequeue(client->used_ring);
        phys = get_phys_addr(buf);
        enqueue(driver->used_ring, phys);

        enqueued += 1;
        /* if a higher priority client has since 
            made a request, go again */
        if (!any_ring_empty(clients[:client])) break;
      }
    }

    if (old_enqueued == new_enqueued) break;
  }
}
\end{minted}
\caption{Transmit Priority-based Policy}
\label{l:prioritybased}
\end{figure}

\subsection{Bandwidth-limited Policy}\label{s:bandwidth}

We implement a bandwidth-limited policy in \autoref{l:bandwidth}. The multiplexer will 
process client transmit requests, calculating how much bandwidth the client has used based on
each request size in a given time window. Should a client exceed its limit within the time window,
the multiplexer will request a time out from the timer driver for the remaining time. As the multiplexer must
regularly communicate with the timer driver to get the time and set timeouts, we design a simple timer driver
API as shown in \autoref{l:timer}. 

\begin{figure} [H]
    \begin{minted}[fontsize=\small]{c}
void process_transmit_ready() {
  curr_time = get_time();
  for each client {
    if (curr_time - client->last >= TIME_WNDW) {
      client->last = curr_time;
      client->bandwidth = 0;
    }
    
    while (!ring_empty(client->used_ring) && 
           !ring_full(driver->used_ring) && 
           client->bandwidth < client->max) {

      buf = dequeue(client->used_ring);
      phys = get_phys_addr(buf);
      enqueue(driver->used_ring, phys);
      /* recalculate the clients bandwidth used in 
        the current time period */ 
      client->bandwidth = calculate_bandwidth(buf_size);
    }

    if (!ring_empty(client->used_ring) && !client->timeout) {
      set_timeout(TIME_WNDW - curr_time - client->last);
      client->timeout = true;
    }
  }
}
\end{minted}
\caption{Transmit Bandwidth-limited Policy}
\label{l:bandwidth}
\end{figure}

\begin{figure} [H]
    \begin{minted}[fontsize=\small]{c}
uint64_t get_time() {
  microkit_ppcall(TIMER, microkit_msginfo_new(GET_TIME, 0));
  return microkit_mr_get(0);
}
        
void set_timeout(uint64_t micros) {
  microkit_mr_set(0, micros);
  microkit_ppcall(TIMER, microkit_msginfo_new(SET_TIMEOUT, 1));
}
\end{minted}
\caption{Timer Driver API}
\label{l:timer}
\end{figure}

The timer driver is a high priority, passive server. It receives interrupts from the device,
and forwards these by notifying the appropriate client or multiplexer. The bandwidth-limited
multiplexer reacts to such notifications by clearing the appropriate client's timeout and
recalling process\_transmit\_ready. Note that this implementation could be made more efficient
by mapping the timer registers read-only into the multiplexer's address space. This would remove
the need for a frequent system call to get the time. However, these implementations are 
meant only as example prototype rather than a prescribed implementation and further optimisations
have been left for use-case specific implementations. \\

This bandwidth-limited policy can also be combined with the round-robin or priority based policy
to order each clients requests. Such a case would be useful in reducing the 
transmit latencies of latency sensitive applications.\\ 

\section{ARP Component}

We implement a separate ARP component that is responsible for handling all broadcast traffic. 
As client MAC addresses are assigned at system design time, our ARP component has a static
mapping of client IDs to virtualised MAC addresses. For client IDs, we use the clients 
communication channel ID as assigned by microkit. 

\subsection{Client Interface}
As ARP is based on both MAC addresses and IPv4 addresses, the new component
requires an interface with client applications. This interface will allow clients
to register a new IPv4 address, change an IPv4 address, or remove one. As adding,
changing and removing an IPv4 address is typically very infrequent for networking systems
and requires minimal data exchange, we define this interface using a protected procedure
call (PPC) from the client to the ARP component. The ARP component can use both the
client ID and the MAC address provided in the arguments to store the clients IPv4 address. 

\begin{figure} [H]
    \begin{minted}[fontsize=\small]{c}
void register_ip4(new_ip4_addr, mac_addr[6])
{
  /* split the MAC address across two registers so it fits */
  microkit_mr_set(0, mac_addr[0:4]);
  microkit_mr_set(1, mac_addr[4:6]);
  microkit_mr_set(2, new_ip4_addr);
  microkit_ppcall(ARP, microkit_msginfo_new(REGISTER_IP, 3));
}

void change_ip4(old_ip4_addr, new_ip4_addr, mac_addr[6])
{
  microkit_mr_set(0, mac_addr[0:4]);
  microkit_mr_set(1, mac_addr[4:6]);
  microkit_mr_set(2, old_ip4_addr);
  microkit_mr_set(3, new_ip4_addr);
  microkit_ppcall(ARP, microkit_msginfo_new(CHANGE_IP, 4));
}  

void remove_ip4(old_ip4_addr, mac_addr[6])
{
  microkit_mr_set(0, mac_addr[0:4]);
  microkit_mr_set(1, mac_addr[4:6]);
  microkit_mr_set(2, old_ip4_addr);
  microkit_ppcall(ARP, microkit_msginfo_new(REMOVE_IP, 3));
}
\end{minted}
\caption{Client Interface to ARP Component}
\label{l:arpintf}
\end{figure}

Currently, the ARP component can only store a single IPv4 address per client, which
is enough for our testing purposes, however this limit can be extended easily in the code
base should the need arise.\\
