\chapter{Conclusion}\label{ch:conclusion}

Typical networking based systems face major problems with either security, performance or both. 
This thesis evaluated a novel framework in the seL4 Device Driver Framework, and expanded it's 
functionality to flexibly support multiple client applications on multi-core systems. We explored
and evaluated how system designers can reinforce different policies using different multiplexer implementations,
adjusting client facing queues, configuring client scheduling parameters, and pinning components across cores.
These parameters can be easily configured for a system's particular use-case, where analysis is required
on that use-case to properly determine the optimal configuration.
We also evaluated our system against an untrusted client application, and 
presented an additional component that protects the rest of the framework from being compromised. 
This new component introduces acceptable overheads, and can be easily configured for a systems threat scenario. 
We conclude that our componentised, microkernel-based framework not only outperforms a typical monolithic
set up in Linux but also provides stronger security guarantees to historically bug prone code. Our static architecture
scales efficiently and flexibly supports multi-client systems. \\

\section{Future Work}

The seL4 Device Driver Framework is not yet a fully functional networking framework as there are a number of features
not yet supported. 
Firstly, our evaluation was confined to UDP only. TCP is a more complex protocol, requiring outgoing packets to be acknowledged 
by the receiver. This means the IP stack must temporarily store buffers until an acknowledgement message is received to enable missed packets 
to be resent. As such, lwIP does not flexibly support zero-copy TCP and changes to the IP stack would be required to properly
support this if lwIP is to be used with our framework. Additionally, to support TCP, further consideration must also be taken
for client initiated TCP transmit using the sDDF as the client must poll incoming queues to receive acknowledgement packets. 
This thesis has omitted a deeper evaluation of TCP and left it for future work. \\

Our performance comparisons against Linux systems were limited to the synchronous socket API. Linux also supports a number
of asynchronous designs, as discussed in \autoref{ch:related_work}, which would provide an interesting comparison against 
our work. Obtaining these would require porting the framework to an x86 architecture with a 1Gbps NIC. \\

This thesis was confined to static architectures. The sDDF aims to support minimal dynamicism by enabling the stopping and restarting
of components as well dynamically changing policies. For example, this thesis could be extended to permit a trusted component
to detect a misbehaving client and restart the client application. Multiplexer components could be swapped out at run time to enable
swappable policy enforcement, however the framework as it stands does not support this and further work is required in measuring 
the performance cost of such a feature. \\

Our evaluation of multi-core systems revealed significant overhead in the SMP configuration of seL4. This warrants a thorough investigation
to properly understand and optimise these overheads. \\

Finally, although our design goals enable our solution to be formally verified, this was left out of scope of this thesis.
The TCB is significantly smaller in our design compared to a monolithic system, but the microkernel, our driver, multiplexers 
and copy components are all trusted. seL4 is already formally verified, however, a bug in one of our other trusted components 
could potentially compromise the system. The simplicity of each of these components means they are accessible to automatic
verification methods. Furthermore, the communication protocols between components means a missed notification could deadlock
the entire system. Fortunately, these protocols can be verified for the absence of such a case using model checking. \\
