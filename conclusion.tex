\chapter{Conclusion}\label{ch:conclusion}

Traditional networking systems often grapple with significant challenges related to security, performance, or a 
combination of both. This thesis conducted an evaluation of the seL4 Device Driver Framework for networking systems,
expanding its functionality to flexibly support multiple client applications on multi-core systems. The highly
modularized design of the sDDF not only surpasses the performance of traditional monolithic setups in Linux,
despite the number of context switches required, but also provides strong fault containment for historically
bug-prone code. This is furthered by the intentional simplicity of each component, minimizing the debugging effort
required by developers and opening the potential for formal verification. \\

This thesis delved into the ways in
which system designers can impose diverse policies through varied, plug-compatible multiplexer implementations,
configuration of client-facing queues, adjustment of client scheduling parameters, and strategic pinning of
components across cores. The example systems we evaluated provide a strong starting point when tailoring these
parameters to configure a system optimally for its particular use case. \\

System designers should also consider the
performance trade-offs when contemplating a single or multi-core system. Our evaluation unveiled substantial
overheads associated with the SMP configuration of seL4, a factor that may be acceptable for compute-intensive
applications. Despite this, we demonstrated that the design can be arbitrarily distributed across cores to cater
to systems when a single CPU is insufficient. \\

Furthermore, we conducted a security analysis to assess potential
vulnerabilities in our design when faced with an untrusted client application. In response, we introduced an
additional component to safeguard the subsystem from potential compromise. Despite the introduced overheads,
this component can be easily configured to suit a particular system's threat scenario and performance
requirements. \\

In conclusion, our evaluation and extension of the seL4 Device Driver Framework debunks
the performance myths surrounding microkernel-based designs, demonstrating performance surpassing traditional
monolithic setups. Significantly, such a design also provides stronger security guarrantees while proving itself 
adaptable for varying use cases.

\section{Future Work}

The seL4 Device Driver Framework is not yet a fully functional networking framework as there are a number of features
not yet supported.
Firstly, our evaluation was confined to UDP only. TCP is a more complex protocol, requiring outgoing packets to be acknowledged 
by the receiver. This means the IP stack must temporarily store buffers until an acknowledgement message is received to enable missed packets 
to be resent. As such, lwIP does not flexibly support zero-copy TCP and changes to the IP stack would be required to properly
support this if lwIP is to be used with our framework. Additionally, to support TCP, further consideration must also be taken
for client initiated TCP transmit using the sDDF as the client must poll incoming queues to receive acknowledgement packets. 
This thesis has omitted a deeper evaluation of TCP and left it for future work. \\

Our performance comparisons against Linux systems were limited to the synchronous socket API. Linux also supports a number
of asynchronous designs, as discussed in \autoref{ch:related_work}, which would provide an interesting comparison against 
our work. Obtaining these would require porting the framework to an x86 architecture with a 10Gbps NIC. \\

This thesis was confined to static architectures. The sDDF aims to support minimal dynamicism by enabling the stopping and restarting
of components as well dynamically changing policies. For example, this thesis could be extended to permit a trusted component
to detect a misbehaving client and restart the client application. Multiplexer components could be swapped out at run time to enable
swappable policy enforcement, however the framework as it stands does not support this and further work is required in measuring 
the performance cost of such a feature. \\

Additionally, our multiplexers were implemented at layer one, though multiplexing at layer two or three may be more practical for some
systems as it would remove the onus on the system designer to set up hardware and protocol addresses for all clients, as well as the 
need for the ARP component. A layer two implementation would require a separate, trusted DHCP client that can resolve DHCP for all
client applications and thus ensure no address collision on the network. A layer three implementation would mean each client uses the 
same protocol and shares an IP address, but is confined to a set of ports. A separate DHCP client would still be required to assign
and manage the shared IP address. \\

Our evaluation of multi-core systems revealed significant overhead in the SMP configuration of seL4. This warrants a thorough investigation
to properly understand and optimise these overheads. \\

Finally, although our design goals enable our solution to be formally verified, this was left out of scope of this thesis.
The TCB is significantly smaller in our design compared to a monolithic system, but the microkernel, our driver, multiplexers 
and copy components are all trusted. seL4 is already formally verified, however, a bug in one of our other trusted components 
could potentially compromise the system. The simplicity of each of these components means they are accessible to automatic
verification methods. Furthermore, the communication protocols between components means a missed notification could deadlock
the entire system. Fortunately, these protocols can be verified for the absence of such a case using model checking. \\
